\documentclass{article}\usepackage{noweb}\pagestyle{noweb}\noweboptions{}\begin{document}\nwfilename{rtest.nw}\nwbegindocs{0}%*   ****************************************************************% ===> this file was generated automatically by noweave --- better not edit it
%*   * (c) Copyright 1997, 1998, 1999, 1900 The Regents of the      *
%*   *     University  of California.  All Rights Reserved.         *
%*   *                                                              *
%*   *     This work was produced under the sponsorship of the      *
%*   *     United States Department of Energy.  The Government      *
%*   *     retains certain rights therein.                          *
%*   ****************************************************************
%*   
%*   ****************************************************************
%*   This file uses the noweb utilities to generate the source code
%*   and documentation.
%*   To generate the LaTex file use:
%*   noweave -index rtest.nw rtest_support.nw > rtest.tex
%*   latex rtest.tex; latex rtest.tex
%*   To generate the PDF version use:
%*   dvipdfm rtest.dvi
%*   To generate the HTML file use:
%*   noweave -filter l2h -index -autodefs c -html 
%*           rtest.nw rtest_support.nw > rtest.html
%*   To produce the source code use:
%*   notangle -L'%N' rtest.nw rtest_support.nw > rtest.py
%*      note: the first empty line must be removed
%*   notangle -L'%N' rtest_init.nw > rtest_init.py
%*   ****************************************************************

\title{Specialized Regression Testing for Griz2(Taurus) verses 
Griz4(Taurus and Mili) Codes}
\author{Victor M. Castillo\\Methods Development Group}
\maketitle

\section{Introduction}
The new version of GRIZ (GRIZ4) has the capability of reading the Mili
database format. For backward compatability, it retains the ability to 
read the Taurus database. 
This specialized regression test is used to compare the results of GRIZ2
and GRIZ4 using the Taurus database. A second comparison is then made 
using GRIZ4 to compare the results of reading the Taurus and Mili databases.\\
\newline
%
\begin{tabular}{|p{1.5cm}|p{2.5cm}|p{1.5cm}|p{2.5cm}|p{1.5cm}|}
 GRIZ2 (Taurus)&  --- compare --- &GRIZ4 (Taurus)&  --- compare --- &GRIZ4 (Mili)\\
\end{tabular}\\
%
\newline
Comparisons are made for a variety of historic sample cases,
including those using the SAND model. The set of sample cases used
in this test are referred to as 
\{{\Tt{}SAMP1}, {\Tt{}SAMP2}, {\Tt{}SAMP4}, {\Tt{}SAMP6}, {\Tt{}SAMP8}, 
{\Tt{}SND1}, {\Tt{}SND2}, {\Tt{}SND3}\}.
For each of these sample cases, derived and primitive results are
compared for each element type available. In addition, a time history
of one of the results (preferably a derived result) is compared for
each element type (see Table 1).

%%%%%%%%%%%%%%%%%%%%%%%%%%%%%%%%%%%%%%%%%%%%%%%%%%%%%%%%%%%%%%%%%%
\begin{table}[h] \centering 
\caption[]{Tested results for each element type}
\begin{tabular}{|p{2.5cm}|p{1.62cm}|p{1.5cm}|p{1.5cm}|p{1.1cm}|p{1.1cm}|p{1.1cm}|}
\hline
Result & Nodal & Shell & Brick & Shared or Global & Material\\
\hline
Derived&
'dispx' 'dispy' 'dispz' 'dispmag' 'velx' 'vely' 'velz' 
'velmag' 'accx' 'accy' 'accz' 'accmag' 'pvmag'
&
'surf1' 'surf2' 'surf3' 'surf4' 'surf5' 'surf6'
'eff1' 'eff2' 'effmax'
&
'ex' 'ey' 'ez' 'exy' 'eyz' 'ezx' 'pdstrn1' 'pdstrn2' 
'pdstrn3' 'pshrstr' 'pstrn1' 'pstrn2' 'pstrn3' 'relvol' 'evol'
&
'sx' 'sy' 'sz' 'sxy' 'syz' 'szx' 'press' 'seff' 'pdev1' 'pdev2' 
'pdev3' 'maxshr' 'prin1' 'prin2' 'prin3'
&\\
\hline
Primal&'nodpos[ux]'&'eeff\_mid'&'eeff'&'ke'&'matpe'\\
\hline
Time History&'dispmag'&'effmax'&'prin1'&'ke'&'matpe'\\
\hline
\end{tabular}
\end{table}
%%%%%%%%%%%%%%%%%%%%%%%%%%%%%%%%%%%%%%%%%%%%%%%%%%%%%%%%%%%%%%%%%%
%\bigskip
%\newline

\subsection{Using the regression test}
The operation of the regression test relies on a particular directory 
structure and a set of particular filenames, most of which are created
by the program if needed. For each sample case (listed in the {\Tt{}CASE} 
dictionary), a directory must be created by the user which contains
the Tuarus and Mili database for the {\Tt{}dyna3d} runs. 
The subdirectories {\Tt{}G2T}, {\Tt{}G4T}, and {\Tt{}G4M} are created within each 
of these case directories to store the results of the GRIZ2(Taurus), GRIZ4(Taurus), 
and GRIZ4(Mili) runs. The results are written to a file with the name
convention
\begin{center}
{\Tt{}casedir}+``/"+{\Tt{}testdir}+``/"+{\Tt{}result{\_}type}+{\Tt{}elem{\_}type}+{\Tt{}result}+``.his".
\end{center}
For example, the derived result {\Tt{}velx} for the {\Tt{}Nodal} element
types from GRIZ4 using the Taurus database from {\Tt{}SAMP1} is called 
``SAMP1/G4T/DRNodalvelx.his".
The Unix {\Tt{}diff} operation are done on comparable
result files and stored in the common case directory. These {\Tt{}diff} 
files are named by the convention:
\begin{center}
{\Tt{}casedir}+``/"+{\Tt{}dir1}+{\Tt{}dir2}+{\Tt{}element{\_}type} +``."+{\Tt{}result{\_}type}.
\end{center}
For example, the comparison between GRIZ2(Taurus) and GRIZ4(Taurus) 
for {\Tt{}Brick} element type for all {\Tt{}derived\ results} of {\Tt{}SAMP1}is called 
``SAMP1/G2TG4TBrick.dr".

Within each case directory, a {\Tt{}.info} file is used to indicate to
the regression test what element types are available for testing. This 
file is generated by a separate {\Tt{}Python} script, {\Tt{}rtest{\_}init.py}
(Generated by {\Tt{}rtest{\_}init.nw} -- see {\Tt{}rtest{\_}init.ps|pdf|html|dvi} for more
documentation. If the {\Tt{}.info} the regression test will not continue.

The {\Tt{}Python} script can be modified to select the cases to be used
and the result type(s). For development of the regression routines, 
set {\Tt{}CASELIST=SHORTLIST}. For exhaustive testing, set {\Tt{}CASELIST=LONGLIST}.
Also, for development, the time history ({\Tt{}TH}) comparisons are quickest.
For more detailed testing uncomment the lines in {\Tt{}main} for the {\Tt{}DR} and
{\Tt{}PR} result types.

\subsection{Revision History} 
\emph{Also see headers in individual functions.}
\begin{itemize}
\item 8-Jun-1999  Original version. (Victor M. Castillo)
\item 14-Dec-1999 Converted to noweb form (This document) (VMC)
\end{itemize}

\nwenddocs{}\nwbegindocs{1}\nwdocspar
\subsection{Program flow}
This section describes the flow of the program and is used by
{\Tt{}NoWeb} to construct the source code.
\nwenddocs{}\nwbegincode{2}\sublabel{NWrte8-*-1}\nwmargintag{{\nwtagstyle{}\subpageref{NWrte8-*-1}}}\moddef{*~{\nwtagstyle{}\subpageref{NWrte8-*-1}}}\endmoddef\nwstartdeflinemarkup\nwenddeflinemarkup
#!/usr/bin/env /grdev/regrtest/bin/python
   \LA{}Define global variables and dictionaries~{\nwtagstyle{}\subpageref{NWrte8-Defe-1}}\RA{}
   \LA{}Main~{\nwtagstyle{}\subpageref{NWrte8-Mai4-1}}\RA{}
   \LA{}Read info file~{\nwtagstyle{}\subpageref{NWrte8-ReaE-1}}\RA{}
   \LA{}Make grizinit file~{\nwtagstyle{}\subpageref{NWrte8-MakI-1}}\RA{}
   \LA{}Griz it~{\nwtagstyle{}\subpageref{NWrte8-Gri7-1}}\RA{}
   \LA{}Onespace~{\nwtagstyle{}\subpageref{NWrte8-One8-1}}\RA{}
   \LA{}Diff it~{\nwtagstyle{}\subpageref{NWrte8-Dif7-1}}\RA{}
   \LA{}Rprint~{\nwtagstyle{}\subpageref{NWrte8-Rpr6-1}}\RA{}
   \LA{}Exit code~{\nwtagstyle{}\subpageref{NWrte8-Exi9-1}}\RA{}

\nwnotused{*}\nwendcode{}\nwbegindocs{3}\nwdocspar
\section{Code listing and description}
Each code section is listed below with a brief description. 

\nwenddocs{}\nwbegindocs{4}\nwdocspar
\subsection{Global variables and dictionaries}
Most of the global variables and dictionaries are defined in
the file {\Tt{}rtest{\_}support.py} (see Appendix). Here {\Tt{}CASELIST}
and {\Tt{}verbose} are defined for development convenience.

\nwenddocs{}\nwbegincode{5}\sublabel{NWrte8-Defe-1}\nwmargintag{{\nwtagstyle{}\subpageref{NWrte8-Defe-1}}}\moddef{Define global variables and dictionaries~{\nwtagstyle{}\subpageref{NWrte8-Defe-1}}}\endmoddef\nwstartdeflinemarkup\nwusesondefline{\\{NWrte8-*-1}}\nwenddeflinemarkup
\LA{}rtest support~{\nwtagstyle{}\subpageref{NWrteG-rteD-1}}\RA{}
#from rtest_support import * #Not used for NoWeb version
SHORTLIST=CASE['SAMP'][0],
LONGLIST=CASE['SAMP']+CASE['SND']
CASELIST=SHORTLIST
verbose=1
clean=0

\nwused{\\{NWrte8-*-1}}\nwendcode{}\nwbegindocs{6}\nwdocspar
\subsection{Main}
Function {\Tt{}main} calls the function {\Tt{}readlist} to get the element
type list appropriate for each sample in {\Tt{}CASELIST}. For each
result type, main calls {\Tt{}griz{\_}it} to do the rest.

\nwenddocs{}\nwbegincode{7}\sublabel{NWrte8-Mai4-1}\nwmargintag{{\nwtagstyle{}\subpageref{NWrte8-Mai4-1}}}\moddef{Main~{\nwtagstyle{}\subpageref{NWrte8-Mai4-1}}}\endmoddef\nwstartdeflinemarkup\nwusesondefline{\\{NWrte8-*-1}}\nwenddeflinemarkup
def main():
   elem_type_list=[]
   for sample in CASELIST:
      os.chdir(RHOME+sample+"/")
      elem_type_list=read_info()
      print sample,": ",elem_type_list 
      #ok=griz_it(sample,'DR',elem_type_list)
      ok=griz_it(sample,'TH',elem_type_list)
      #ok=griz_it(sample,'PR',elem_type_list)
   return 0

\nwused{\\{NWrte8-*-1}}\nwendcode{}\nwbegindocs{8}\nwdocspar
\subsection{Parse the {\Tt{}.info} file}
The {\Tt{}.info} file, created by {\Tt{}rtest{\_}init}, is parsed to determine
what element type(s) are available for testing. 

\nwenddocs{}\nwbegincode{9}\sublabel{NWrte8-ReaE-1}\nwmargintag{{\nwtagstyle{}\subpageref{NWrte8-ReaE-1}}}\moddef{Read info file~{\nwtagstyle{}\subpageref{NWrte8-ReaE-1}}}\endmoddef\nwstartdeflinemarkup\nwusesondefline{\\{NWrte8-*-1}}\nwenddeflinemarkup
def read_info(): 
   "Parse ./.info file to determine relevant element types"
   if not os.path.exists('.info'):
      print("ERROR: No info file found. Run rtest_init.py.")
      return -1
   info=open('.info','r+')
   raw = info.read() 
   info.close()
   keylist=[]
   for key in IS.keys():
      etest = re.compile(IS[key][0])
      if etest.search(raw):
         keylist.append(key)
   return keylist

\nwused{\\{NWrte8-*-1}}\nwendcode{}\nwbegindocs{10}\nwdocspar
\subsection{Griz loop}
The function {\Tt{}griz{\_}it} does the bulk of the work of the regression
test. For each element type, {\Tt{}make{\_}g{\_}file} is called to generate 
a grizinit file. The appropriate GRIZ binary is then executed, generating
result data files. Care is taken to make sure that the GRIZ process is
finished before comparing result data files by checking for the file 
named "{\Tt{}done}" (created by GRIZ).
The function {\Tt{}diff{\_}it} is then called to compare the result data files.

\nwenddocs{}\nwbegincode{11}\sublabel{NWrte8-Gri7-1}\nwmargintag{{\nwtagstyle{}\subpageref{NWrte8-Gri7-1}}}\moddef{Griz it~{\nwtagstyle{}\subpageref{NWrte8-Gri7-1}}}\endmoddef\nwstartdeflinemarkup\nwusesondefline{\\{NWrte8-*-1}}\nwenddeflinemarkup
def griz_it(sample,result_type,elem_type_list):
   result_type_string=RS[result_type]
   os.chdir(RHOME+sample)
   for elem_type in elem_type_list:
      rprint("Testing "+elem_type+" "+result_type_string+" with "+sample)
      for i in range(len(TEST['bin'])):
         GRIZBIN=TEST['bin'][i]   #selects the binary, Griz2 or Griz4
         DB=TEST['db'][i]         #selects the database, Taurus or Mili
         DIR=TEST['dir'][i]       #selects the appropriate directory for results
         if not os.path.exists(DIR):
            # if appropriate directory does not exist, create it
            if verbose:print "creating directory: "+DIR
            os.mkdir(DIR)
         os.chdir(DIR)              #change to appropriate directory
         ok=make_g_file(result_type, elem_type)   #create grizinit file 
         err=os.system(GRIZBIN+" -i "+DB)   #run griz with new grizinit file
         os.chdir('..')
      #force all runs to finish before comparing results
      for i in [0,1,2]:
         DIR=TEST['dir'][i]
         if not os.path.exists(DIR):
            print("ERROR: Directory "+DIR+"was not created")
         if not os.path.exists(DIR+'/done'):
            print("waiting for "+DIR+"...")
         while not os.path.exists(DIR+'/done'):
            pass
      for i in [0,1,2]:
         os.remove(TEST['dir'][i]+'/done')
      ok=diff_it(result_type,elem_type)
   rprint("fini")
   return 1

\nwused{\\{NWrte8-*-1}}\nwendcode{}\nwbegindocs{12}\nwdocspar
\subsection{Make {\Tt{}grizinit} file}
Here the {\Tt{}grizinit} is created for each result type.
Each {\Tt{}grizinit} file instructs GRIZ to run the sample case
for all results associated with each element type available.
A GRIZ history file of the name 
{\Tt{}result{\_}type}+{\Tt{}elem{\_}type}+{\Tt{}result}+``.his"
is created by GRIZ for the result data.

\nwenddocs{}\nwbegincode{13}\sublabel{NWrte8-MakI-1}\nwmargintag{{\nwtagstyle{}\subpageref{NWrte8-MakI-1}}}\moddef{Make grizinit file~{\nwtagstyle{}\subpageref{NWrte8-MakI-1}}}\endmoddef\nwstartdeflinemarkup\nwusesondefline{\\{NWrte8-*-1}}\nwenddeflinemarkup
def make_g_file(result_type,elem_type):
   g=open('grizinit','w+')    #create grizinit file 
   g.write(HEADER)
   if result_type=='DR':
      for result in DR[elem_type]: 
         hisfile='DR'+elem_type+result+".his"
         g.write("show "+result+"\\n")
         g.write("savtxt "+hisfile+"\\n")
         g.write("tellmm\\n")
         g.write("endtxt\\n")
      g.write("savtxt done\\n")
      g.write("endtxt\\n")
   elif result_type=='PR':
      for result in PR[elem_type]: 
         hisfile='PR'+elem_type+result+".his"
         g.write("show "+result+"\\n")
         g.write("savtxt "+hisfile+"\\n")
         g.write("tellmm\\n")
         g.write("endtxt\\n")
      g.write("savtxt done\\n")
      g.write("endtxt\\n")
   elif result_type=='TH':
      for result in TH[elem_type]: 
         hisfile='TH'+elem_type+result+".his"
         g.write("select node 1\\n")
         g.write("select beam 1\\n")
         g.write("select shell 1\\n")
         g.write("select brick 1\\n")
         g.write("show "+result+"\\n")
         g.write("timhis \\n")
         g.write("outth "+hisfile+"\\n")
      g.write("savtxt done\\n")
      g.write("endtxt\\n")
   else:
      g.write("quit\\n")
      g.close()
      return 0
   g.write("quit\\n")
   g.close()
   return 1

\nwused{\\{NWrte8-*-1}}\nwendcode{}\nwbegindocs{14}\nwdocspar
\subsection{Compare the results}
This routine uses the Unix {\Tt{}diff} command to compare result files. 
the difference is piped to a file in the sample case directory.

\nwenddocs{}\nwbegincode{15}\sublabel{NWrte8-Dif7-1}\nwmargintag{{\nwtagstyle{}\subpageref{NWrte8-Dif7-1}}}\moddef{Diff it~{\nwtagstyle{}\subpageref{NWrte8-Dif7-1}}}\endmoddef\nwstartdeflinemarkup\nwusesondefline{\\{NWrte8-*-1}}\nwenddeflinemarkup
def diff_it(result_type,elem_type):
   # compare results for G2T and G4T and G4M
   for i in [0,1]:
      DIR1=TEST['dir'][i]
      DIR2=TEST['dir'][i+1]
      rprint("Comparing results: "+DIR1+" and "+DIR2)
      ext="."+string.lower(result_type)
      TEMP=DIR1+DIR2+elem_type+ext
      err=os.system('echo  >'+TEMP)
      for result in eval(result_type)[elem_type]:
         TARG1=DIR1+'/'+result_type+elem_type+result+".his"
         TARG2=DIR2+'/'+result_type+elem_type+result+".his"
         if result_type =='TH':
            ok=onespace(TARG1)
            ok=onespace(TARG2)
         err=os.system('diff '+TARG1+' '+TARG2+'>>'+TEMP)
   for i in [0,1,2]:
      for result in eval(result_type)[elem_type]:
         hisfile=TEST['dir'][i]+'/'+result_type+elem_type+result+".his"
         if not os.path.exists(hisfile):
            msg="File not created: "+hisfile
            rprint(msg)
            err=os.system('echo '+msg+' >>'+TEMP)
         else:
            if clean:
               err=os.remove(hisfile)
   return 1

\nwused{\\{NWrte8-*-1}}\nwendcode{}\nwbegindocs{16}\nwdocspar
\subsection{Rprint}
This routine sends messages to the log file and to the standard
output if verbose. I might add a ranking to each message so that 
it is printed to standard output only if it is below the "verbosity"
number.

\nwenddocs{}\nwbegincode{17}\sublabel{NWrte8-Rpr6-1}\nwmargintag{{\nwtagstyle{}\subpageref{NWrte8-Rpr6-1}}}\moddef{Rprint~{\nwtagstyle{}\subpageref{NWrte8-Rpr6-1}}}\endmoddef\nwstartdeflinemarkup\nwusesondefline{\\{NWrte8-*-1}}\nwenddeflinemarkup
def rprint(text):
   "Reports information to report file and console if verbose"
   # r.write(text+"\\n")
   if verbose: print(text) 
   return 0

\nwused{\\{NWrte8-*-1}}\nwendcode{}\nwbegindocs{18}\nwdocspar
\subsection{Onespace}
This routine takes the GRIZ result files and converts it
to one where the data is delimited by a single space. This 
is needed in cases where the versions have different formatting.

\nwenddocs{}\nwbegincode{19}\sublabel{NWrte8-One8-1}\nwmargintag{{\nwtagstyle{}\subpageref{NWrte8-One8-1}}}\moddef{Onespace~{\nwtagstyle{}\subpageref{NWrte8-One8-1}}}\endmoddef\nwstartdeflinemarkup\nwusesondefline{\\{NWrte8-*-1}}\nwenddeflinemarkup
def onespace(filename):
   tempfilename='tempfile'
   tempfile=open(tempfilename,'w+')
   datafile=open(filename,'r+')
   datin=datafile.readlines()
   for i in range(len(datin)):
      j=string.join(string.split(datin[i]))
      tempfile.write(j+"\\n")
   datafile.close()
   tempfile.close()
   err=os.system('mv '+tempfilename+' '+filename)
   return 1

\nwused{\\{NWrte8-*-1}}\nwendcode{}\nwbegindocs{20}\nwdocspar
\subsection{Exit}
Normal system exits come from Main.
\nwenddocs{}\nwbegincode{21}\sublabel{NWrte8-Exi9-1}\nwmargintag{{\nwtagstyle{}\subpageref{NWrte8-Exi9-1}}}\moddef{Exit code~{\nwtagstyle{}\subpageref{NWrte8-Exi9-1}}}\endmoddef\nwstartdeflinemarkup\nwusesondefline{\\{NWrte8-*-1}}\nwenddeflinemarkup
if __name__ == '__main__':
   sys.exit(main())



\nwused{\\{NWrte8-*-1}}\nwendcode{}\nwfilename{rtest_support.nw}\nwbegindocs{0}%*   ****************************************************************
%*   * (c) Copyright 1997, 1998, 1999, 1900 The Regents of the      *
%*   *     University  of California.  All Rights Reserved.         *
%*   *                                                              *
%*   *     This work was produced under the sponsorship of the      *
%*   *     United States Department of Energy.  The Government      *
%*   *     retains certain rights therein.                          *
%*   ****************************************************************
%*   
%*   ****************************************************************
%*   This file uses the noweb utilities to generate the source code
%*   and documentation.
%*   To generate the LaTex file use:
%*   noweave -index rtest_init.nw > rtest_init.tex
%*   To generate the HTML file use:
%*   noweave -filter l2h -index -autodefs c -html 
%*           rtest_init.nw > rtest_init.html
%*   ****************************************************************


\nwenddocs{}\nwbegindocs{1}\nwdocspar
\section{Appendix: {\Tt{}rtest{\_}support}}
This file defines all of the Python dictionaries used for the 
regression testing. Dictionaries have an unordered set of 
\emph{key:value} pairs where the key can be a string.

\nwenddocs{}\nwbegincode{2}\sublabel{NWrteG-rteD-1}\nwmargintag{{\nwtagstyle{}\subpageref{NWrteG-rteD-1}}}\moddef{rtest support~{\nwtagstyle{}\subpageref{NWrteG-rteD-1}}}\endmoddef\nwstartdeflinemarkup\nwusesondefline{\\{NWrte8-Defe-1}}\nwenddeflinemarkup

import os,sys,re,string

verbose=1
RHOME="/grdev/regrtest/"
GRIZ2=RHOME+"bin/griz2"
GRIZ4=RHOME+"bin/griz4"
TAURUS="../d3plot"
MILI="../m_plot"
H1="# This grizinit file was automatically generated\\n"
H2="# by regrtest.py for regression testing of Griz4\\n"
H3="# -Vic Castillo\\n#\\n"
HEADER=H1+H2+H3

TEST=\{\}
TEST['bin']=GRIZ2,GRIZ4,GRIZ4
TEST['db']=TAURUS,TAURUS,MILI
TEST['dir']='G2T','G4T','G4M'

CASE=\{\}
CASE['SAMP']='SAMP1','SAMP2','SAMP4','SAMP6','SAMP8'
CASE['SND']='SND1','SND2','SND3'

# Derived Results Dictionary
DR=\{\}
drnd1='dispx','dispy','dispz','dispmag'
drnd2='velx','vely','velz','velmag'
drnd3='accx','accy','accz','accmag','pvmag'
DR['Nodal']=drnd1+drnd2+drnd3
drshr1='sx','sy','sz','sxy','syz','szx'
drshr2='press','seff','pdev1','pdev2','pdev3'
drshr3='maxshr','prin1','prin2','prin3'
DR['Share']=drshr1+drshr2+drshr3
drshl1='surf1','surf2','surf3','surf4','surf5','surf6'
drshl2='eff1','eff2','effmax'
DR['Shell']=drshl1+drshl2
drbrk1='ex','ey','ez','exy','eyz','ezx'
drbrk2='pdstrn1','pdstrn2','pdstrn3','pshrstr'
drbrk3='pstrn1','pstrn2','pstrn3','relvol','evol'
DR['Brick']=drbrk1+drbrk2+drbrk3

# Primal Results Dictionary
PR=\{\}
PR['Nodal']='nodpos[ux]',
PR['Global']='ke',
PR['Mat']='matpe',
PR['Brick']='eeff',
PR['Shell']='eeff_mid',

# Time History Dictionary
# These are the results selected for the time histories
TH=\{\}
TH['Nodal']='dispmag',
TH['Global']='ke',
TH['Mat']='matpe',
TH['Brick']='prin1',
TH['Shell']='effmax',

# Info String Dictionary
# These are the string patterns from the griz 'info' command.
# (Why does it not work without the commas?)
IS=\{\}
IS['Nodal']='Nodes:',
IS['Brick']='Hex elements:',
IS['Shell']='Shell elements:',
# IS['Beam']='Beam elements:',

# These are the strings describing result types
RS=\{\}
RS['DR']='Derived Results'
RS['PR']='Primative Results'
RS['TH']='Time History Results'


\nwused{\\{NWrte8-Defe-1}}\nwendcode{}

\nwixlogsorted{c}{{*}{NWrte8-*-1}{\nwixd{NWrte8-*-1}}}%
\nwixlogsorted{c}{{Define global variables and dictionaries}{NWrte8-Defe-1}{\nwixu{NWrte8-*-1}\nwixd{NWrte8-Defe-1}}}%
\nwixlogsorted{c}{{Diff it}{NWrte8-Dif7-1}{\nwixu{NWrte8-*-1}\nwixd{NWrte8-Dif7-1}}}%
\nwixlogsorted{c}{{Exit code}{NWrte8-Exi9-1}{\nwixu{NWrte8-*-1}\nwixd{NWrte8-Exi9-1}}}%
\nwixlogsorted{c}{{Griz it}{NWrte8-Gri7-1}{\nwixu{NWrte8-*-1}\nwixd{NWrte8-Gri7-1}}}%
\nwixlogsorted{c}{{Main}{NWrte8-Mai4-1}{\nwixu{NWrte8-*-1}\nwixd{NWrte8-Mai4-1}}}%
\nwixlogsorted{c}{{Make grizinit file}{NWrte8-MakI-1}{\nwixu{NWrte8-*-1}\nwixd{NWrte8-MakI-1}}}%
\nwixlogsorted{c}{{Onespace}{NWrte8-One8-1}{\nwixu{NWrte8-*-1}\nwixd{NWrte8-One8-1}}}%
\nwixlogsorted{c}{{Read info file}{NWrte8-ReaE-1}{\nwixu{NWrte8-*-1}\nwixd{NWrte8-ReaE-1}}}%
\nwixlogsorted{c}{{Rprint}{NWrte8-Rpr6-1}{\nwixu{NWrte8-*-1}\nwixd{NWrte8-Rpr6-1}}}%
\nwixlogsorted{c}{{rtest support}{NWrteG-rteD-1}{\nwixu{NWrte8-Defe-1}\nwixd{NWrteG-rteD-1}}}%
\end{document}

